\documentclass{article}
\usepackage[margin=0.75in]{geometry}
\usepackage{fancyhdr}
\pagestyle{fancy}
\lhead{}
\chead{}
\rhead{Warren D. (Hoss) Craft Final Project ECE 595}
\lfoot{}
\cfoot{\thepage}
\rfoot{}
\usepackage{amsmath, amsthm, amssymb}
\usepackage{mathtools}
\usepackage{bbold} % allow an identity elem 1
\usepackage{graphicx}
\usepackage{caption}
\usepackage{hyperref}
\usepackage{changepage}
\usepackage{lipsum}
\usepackage{scrextend} % for block indents (for example)
\usepackage{color, colortbl} % For Coloring Table Elements
\usepackage{listings}
\usepackage{makecell}  % allows more complex table formatting
\usepackage{multirow}
\usepackage{proof}     % for inference trees
\usepackage{subcaption}
\usepackage{enumitem}  % allows customization of enumerate

\usepackage{wrapfig}   % allows wrapping text around figures
\usepackage{sidecap}   % allows side captions on figures
\sidecaptionvpos{figure}{t}
\usepackage{physics}   % provides abs{} and norm{}, as well as bra{} and ket{} notations

% for the makecell package:
\renewcommand\theadalign{bc}
\renewcommand\theadfont{\bfseries}
\renewcommand\theadgape{\Gape[4pt]}
\renewcommand\cellgape{\Gape[4pt]}

\addtolength{\hoffset}{-1cm}
\addtolength{\textwidth}{1cm}
\graphicspath{ {IMAGES/} }

\definecolor{Gray}{gray}{0.9}
\newcolumntype{g}{>{\columncolor{Gray}}c}

% for bibliography and citations
\usepackage{biblatex}
\addbibresource{bibliography.bib}

% for script lettering, e.g.
% for Fourier transform
\usepackage[mathscr]{euscript}

% For quantum circuit diagrams
\usepackage{qcircuit}

% For shading paragraph boxes
\usepackage[framemethod=tikz]{mdframed}

\title{ECE 595: Introduction to Quantum Computing\\
Final Project: Review of Quantum Phase Estimation}
\author{Warren D. (Hoss) Craft}
\date{Tues 5/12/2020}

\begin{document}
\maketitle
%\tableofcontents

\flushleft

Note: References throughout the homework to ``Mermin'' refer to Mermin's (2007) \textit{Quantum Computer Science: An Introduction} \cite{Mermin_2007}, and references to ``Nielsen \& Chuang'' or just ``N\&C'' refer to Nielsen \& Chuang's (2010) \textit{Quantum Computation and Quantum Information (10th Anniversary Edition)} \cite{Nielsen_&_Chuang_2010}, and further bracketed citations are omitted.

\begin{enumerate}[label=\textbf{(\arabic*)}]

%%%%%%%%%%%%%%%%%%%%%%%%%%%%%%%%%
%   N&C Exercise 5.7 (pg 222)   %
%%%%%%%%%%%%%%%%%%%%%%%%%%%%%%%%%

\item Nielsen \& Chuang's Exercise 5.7 (pg 222):
\vspace{0.1in}
\begin{adjustwidth}{100pt}{100pt}
\begin{mdframed}[hidealllines=true, backgroundcolor=gray!20]
\textbf{Exercise 5.7:} Additional insight into the circuit in Figure 5.2 may be obtained by showing, as you should now do, that the effect of the sequence of controlled-U operations like that in Figure 5.2 is to take the state $\ket{j}\ket{u}$ to $\ket{j}U^{j}\ket{u}$. (Note that this does not depend on $\ket{u}$ being an eigenstate of $U$.)
\end{mdframed}
\end{adjustwidth}

\textbf{Solution.} We take $\ket{j} = \ket{j}_{t}$ --- \textit{i.e.}, $j$ is an integer value expressible in $t$ bits, so that
\begin{align}
    j = \sum_{i=0}^{t-1} \alpha_i 2^i, \text{ for } \alpha_i \in \{0, 1\}
\end{align}
and thus
\begin{align}
    \ket{j} = \ket{\alpha_{t-1}}\ket{\alpha_{t-1}}\ldots\ket{\alpha_{1}}\ket{\alpha_{0}}
\end{align}
Then the effect of the sequence of controlled-$U$ operations like that in Figure 5.2 is to effect the transformation:
\begin{align}
    \ket{j}\ket{u} 
    &\mapsto
    \ket{j}\prod_{i=0}^{t-1}{e^{2\pi i (\alpha_i 2^i \varphi)}}\ket{u}
\end{align}
because for each non-zero $\alpha_i$ in the expansion of $j$, we will obtain a (non-unitary) factor of $e^{2\pi i (2^i \varphi)}$. The product of the exponential functions can be re-expressed as sum in the exponent, where the sum is exactly our formula for $j$: 
\begin{align}
    \ket{j}\prod_{i=0}^{t-1}{e^{2\pi i (\alpha_i 2^i \varphi)}}\ket{u}
    &=
    \ket{j}{e^{2\pi i (\sum_{i=0}^{t-1}(\alpha_i 2^i) \varphi)}}\ket{u}\\
    &=
    \ket{j}{e^{2\pi i (j \varphi)}}\ket{u}\\
    &=
    \ket{j}{\left(e^{2\pi i \varphi}\right)^j}\ket{u}\\
    &=
    \ket{j}{U^j}\ket{u}\\
\end{align}
where in the end we used the fact that $U\ket{u} = e^{2\pi i \varphi}\ket{u}$ and thus $U^{j}\ket{u} = (e^{2\pi i \varphi})^j\ket{u}$.

Some simple concrete examples of this result are shown worked out from scratch in Table \ref{table:Prob_5.7} for $j$ defined on $t=3$ Qbits.

\begin{table}[th]
    \captionsetup{font=small, width = 14cm}
    \centering
    \begin{tabular}{c|c|c|c}
        $\ket{j}\ket{u}$ input & output & simplified output\\
        \hline\hline
        &&\\[-5pt]
        $\ket{001}\ket{u}$
        & $\ket{001}U^{2^0}\ket{u}$
        & $\ket{001}U^{1}\ket{u} = \ket{1}_3 U^{1}\ket{u}$
        \\[5pt]
        \hline
        &&\\[-5pt]
        $\ket{010}\ket{u}$
        & $\ket{010}U^{2^1}\ket{u}$
        & $\ket{010}U^{2}\ket{u} = \ket{2}_3 U^{2}\ket{u}$
        \\[5pt]
        \hline
        &&\\[-5pt]
        $\ket{011}\ket{u}$
        & $\ket{011}U^{2^1}U^{2^0}\ket{u}$
        & $\ket{011}U^{2}U^{1}\ket{u} = \ket{3}_3 U^{3}\ket{u}$
        \\[5pt]
        \hline
        &&\\[-5pt]
        $\ket{100}\ket{u}$
        & $\ket{100}U^{2^2}\ket{u}$
        & $\ket{100}U^{4}\ket{u} = \ket{4}_3 U^{4}\ket{u}$
        \\[5pt]
        \hline
        &&\\[-5pt]
        $\ket{101}\ket{u}$
        & $\ket{101}U^{2^2}U^{2^0}\ket{u}$
        & $\ket{101}U^{4}U^{1}\ket{u} = \ket{5}_3 U^{5}\ket{u}$
        \\[5pt]
        \hline
        &&\\[-5pt]
        $\ket{110}\ket{u}$
        & $\ket{110}U^{2^2}U^{2^1}\ket{u}$
        & $\ket{110}U^{4}U^{2}\ket{u} = \ket{6}_3 U^{6}\ket{u}$
        \\[5pt]
        \hline
        \end{tabular}
    \caption{Examples of results for Exercise 5.7, with $j$ defined on 3 Qbits.}
    \label{table:Prob_5.7}
\end{table}

\clearpage

% \clearpage
\end{enumerate}

% \clearpage
% ========== %
% END MATTER %
% ========== %
% \newpage
\printbibliography

\end{document}