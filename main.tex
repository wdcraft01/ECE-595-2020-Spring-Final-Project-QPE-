\documentclass{article}
\usepackage[margin=0.75in]{geometry}
\usepackage{fancyhdr}
\pagestyle{fancy}
\lhead{}
\chead{}
\rhead{Warren D. (Hoss) Craft Final Project ECE 595}
\lfoot{}
\cfoot{\thepage}
\rfoot{}
\usepackage{amsmath, amsthm, amssymb}
\usepackage{mathtools}
\usepackage{bbold} % allow an identity elem 1
\usepackage{graphicx}
\usepackage{caption}
\usepackage{hyperref}
\usepackage{changepage}
\usepackage{lipsum}
\usepackage{scrextend} % for block indents (for example)
\usepackage{color, colortbl} % For Coloring Table Elements
\usepackage{listings}
\usepackage{makecell}  % allows more complex table formatting
\usepackage{multirow}
\usepackage{proof}     % for inference trees
\usepackage{subcaption}
\usepackage{enumitem}  % allows customization of enumerate

\usepackage{wrapfig}   % allows wrapping text around figures
\usepackage{sidecap}   % allows side captions on figures
\sidecaptionvpos{figure}{t}
\usepackage{physics}   % provides abs{} and norm{}, as well as bra{} and ket{} notations

% for the makecell package:
\renewcommand\theadalign{bc}
\renewcommand\theadfont{\bfseries}
\renewcommand\theadgape{\Gape[4pt]}
\renewcommand\cellgape{\Gape[4pt]}

\addtolength{\hoffset}{-1cm}
\addtolength{\textwidth}{1cm}
\graphicspath{ {IMAGES/} }

\definecolor{Gray}{gray}{0.9}
\newcolumntype{g}{>{\columncolor{Gray}}c}

% for bibliography and citations
\usepackage{biblatex}
\addbibresource{bibliography.bib}

% for script lettering, e.g.
% for Fourier transform
\usepackage[mathscr]{euscript}

% For quantum circuit diagrams
% \usepackage{qcircuit}
\usepackage{tikz}
\usetikzlibrary{quantikz}

% For shading paragraph boxes
\usepackage[framemethod=tikz]{mdframed}

\title{ECE 595: Introduction to Quantum Computing\\
Final Project: Review of Quantum Phase Estimation}
\author{Warren D. (Hoss) Craft}
\date{Tues 5/12/2020}

\begin{document}
\maketitle
%\tableofcontents

\flushleft

Note: References throughout the homework to ``Mermin'' refer to Mermin's (2007) \textit{Quantum Computer Science: An Introduction} \cite{Mermin_2007}, and references to ``Nielsen \& Chuang'' or just ``N\&C'' refer to Nielsen \& Chuang's (2010) \textit{Quantum Computation and Quantum Information (10th Anniversary Edition)} \cite{Nielsen_&_Chuang_2010}, and further bracketed citations are omitted.

\begin{enumerate}[label=\textbf{(\arabic*)}]

\item Intro

Nielsen \& Chuang (2010) segue from the quantum Fourier transform to quantum phase estimation (QPE) by remarking that ``The Fourier transform is the key to a general procedure known as phase estimation, which in turn is the key for many quantum algorithms'' (pg 221). Among the many quantum algorithms for which QPE is key are the quantum order-finding and quantum factoring algorithms, covered in N\&C \S\S 5.31--5.32 (pp 226--234).

\textbf{\textit{QPE Problem Statement}} (based on N\&C, and the QPE page in \textit{Wikipedia} \cite{wikipedia:qpe}):

\begin{adjustwidth}{0.5in}{0.5in}
Let $U$ be a unitary operator that operates on the $m$-Qbit $\ket{u}$ such that $U\ket{u} = e^{2\pi i \varphi}\ket{u}$ --- \textit{i.e.}, $\ket{u}$ is an eigenstate of the operator $U$ with eigenvalue $e^{2\pi i \varphi}$, where $\varphi$ is an unknown real value with $0\le\varphi<1$.

Goal: estimate $\varphi$ (which is then equivalent to estimating the eigenvalue $e^{2\pi i \varphi}$).
\end{adjustwidth}

\textbf{\textit{The eigenvalue $e^{2\pi i \varphi}$}}

Although the eigenvalue form $e^{2\pi i \varphi}$ at first seems highly specific or idiosyncratic, as pointed out in \cite{wikipedia:qpe}, the eigenvalue $\lambda_u$ associated with eigenstate $\ket{u}$ can always be written in the form $e^{2\pi i \varphi}$ because $U$ is a unitary operator over a complex vector space, so its eigenvalues must be expressible as complex numbers with modulus 1.

More explicitly, suppose $\lambda_{u} = a + bi$ for real values $a$ and $b$. Then $\theta = 2\pi\varphi = \tan^{-1}(\frac{b}{a})$ gives $\varphi = \frac{1}{2\pi}\tan^{-1}(\frac{b}{a})$ (with suitable accommodations for cases where $a=0$ and using the signs of $a$ and $b$ to decide on the correct quadrant for the inverse tangent result) to produce $\lambda_{u} = a + bi = \cos(2\pi\varphi) + i\sin(2\pi\varphi) = e^{2\pi i \varphi}$.

%%%%%%%%%%%%%%%%%%%%%%%%%%%%%%%%%
%   The QPE Procedure           %
%%%%%%%%%%%%%%%%%%%%%%%%%%%%%%%%%
\item The QPE Procedure: Stage 1

\begin{itemize}
  \item[\textbf{(\textit{a})}] Registers and Initial Conditions
  
  \vspace{0.05in}
  
  The QPE algorithm utilizes 2 registers. The 1st register consists of $t$ Qbits, each initially in the state $\ket{0}$. The 2nd register consists of $m$ Qbits and is used to store the eigenstate $\ket{u}$. The circuit for the algorithm is shown in Figure (\ref{fig:QPE_circuit_diagram}) below, which is a reasonable facsimile of N\&C's Figure 5.2 (pg 222), with Qbits going from top to bottom in order of decreasing significance --- \textit{e.g.}, we''' refer to the upper-most Qbit in the 1st register at Qbit $t-1$ and the bottom-most Qbit in the 1st register as Qbit $0$.
  
  \vspace{0.05in}
  
  \item[\textbf{(\textit{b})}] Hadamard applied to 1st register
  
  \vspace{0.05in}
  
  The QPE algorithm begins with what Mermin describes as the ``standard trick'': the application of a Hadamard to each Qbit of the 1st register to produce the uniformly-weighted superposition:
  
  \begin{align}
      H_{t-1}H_{t-2}\ldots H_{1}H_{0}\ket{00\ldots 00}
      &=
      \frac{1}{\sqrt{2}}(\ket{0} + \ket{1})
      \frac{1}{\sqrt{2}}(\ket{0} + \ket{1})
      \ldots
      \frac{1}{\sqrt{2}}(\ket{0} + \ket{1})
      \frac{1}{\sqrt{2}}(\ket{0} + \ket{1})\\
      &=
      \frac{1}{2^{t/2}}\sum_{x=0}^{2^t-1}\ket{x}_{t}
  \end{align}
  
  \vspace{0.05in}
  
  \item[\textbf{(\textit{c})}] Controlled $U^{2^j}$ operations applied to the 2nd register \ldots
  
  \vspace{0.05in}
  
  The QPE algorithm continues with the successive controlled $U^{2^j}$ operations applied to the 2nd register, with each $U^{2^j}$ application being controlled by the $j$\textit{th} Qbit in the 1st register.
  
  \vspace{0.05in}
  
  To understand a bit better the eventual result in this first stage, shown along the right-hand side of Figure (\ref{fig:QPE_circuit_diagram}), consider first the process experienced by just the $0$th Qbit in the first register:
  
  \begin{align}
      C_{0\ket{u}}^{U^{2^0}}H_{0}\ket{0}\ket{u}
      &=
      \frac{1}{\sqrt{2}}C_{0\ket{u}}^{U^{2^0}}(\ket{0} + \ket{1})\ket{u}\\
      &=
      \frac{1}{\sqrt{2}}C_{0\ket{u}}^{U^{2^0}}(\ket{0}\ket{u} + \ket{1}\ket{u})\\
      &=
      \frac{1}{\sqrt{2}}(\ket{0}\ket{u} + \ket{1}U^{2^0}\ket{u})\\
      &=
      \frac{1}{\sqrt{2}}(\ket{0}\ket{u} + \ket{1}e^{2\pi i (2^0 \varphi)}\ket{u})\\
      &=
      \frac{1}{\sqrt{2}}(\ket{0} + e^{2\pi i (2^0 \varphi)}\ket{1})\ket{u}\\
      &=
      \ket{\psi_0}\ket{u}
  \end{align}

where $\ket{\psi_0}$ is the result for Qbit $0$ in the 1st register, shown along the right-hand side of Figure (\ref{fig:QPE_circuit_diagram}).

Then consider the next part of the process for Qbit $1$:

\begin{align}
      C_{1\ket{u}}^{U^{2^1}}H_{1}\ket{0}\ket{\psi_0}\ket{u}
      &=
      \frac{1}{\sqrt{2}}
      C_{1\ket{u}}^{U^{2^1}}(\ket{0}+\ket{1})\ket{\psi_0}\ket{u}\\
      &=
      \frac{1}{\sqrt{2}}
      (\ket{0}\ket{\psi_0}\ket{u}+\ket{1}\ket{\psi_0}U^{2^1}\ket{u})\\
      &=
      \frac{1}{\sqrt{2}}
      (\ket{0}\ket{\psi_0}\ket{u}+\ket{1}\ket{\psi_0}e^{2\pi i (2^1 \varphi)}\ket{u})\\
      &=
      \frac{1}{\sqrt{2}}
      (\ket{0}+e^{2\pi i (2^1 \varphi)}\ket{1})\ket{\psi_0}\ket{u}\\
      &=
      \ket{\psi_1}\ket{\psi_0}\ket{u}
  \end{align}
  
  Eventually, of course, this continued process leads to the 1st-stage result of $\ket{\psi_{t-1}}\ldots\ket{\psi_1}\ket{\psi_0}\ket{u}$.
  


%%%%%%%%%%%%%%%%%%%%%%%%%%%%%%%%%
%   The QPE Circuit Diagram     %
%%%%%%%%%%%%%%%%%%%%%%%%%%%%%%%%%

\begin{figure}[ht]
\captionsetup{font=small, width = 14cm}
\centering
\centerline{
    \begin{quantikz}
    % ROW 1
    \lstick[wires=5]{First Register:\\$t$ Qbits} &
    \lstick{$\ket{0}$} & \gate{H} &
        \qw & \qw & \qw & \ \ldots\ \qw & \ctrl{5} & \rstick{$\frac{1}{\sqrt{2}}(\ket{0}+e^{2\pi i (2^{t-1} \varphi)}\ket{1})=\ket{\psi_{t-1}}$}\qw\\
    % ROW 2
    &\cdots & \cdots &
        \cdots & \cdots & \cdots & \cdots & \cdots & \cdots\\
    % ROW 3
    &\lstick{$\ket{0}$} & \gate{H} &
        \qw & \qw & \ctrl{3} & \ \ldots\ \qw & \qw & \rstick{$\frac{1}{\sqrt{2}}(\ket{0}+e^{2\pi i (2^2 \varphi)}\ket{1})=\ket{\psi_2}$}\qw\\
    % ROW 4
    &\lstick{$\ket{0}$} & \gate{H} &
        \qw & \ctrl{2} & \qw & \ \ldots\ \qw & \qw & \rstick{$\frac{1}{\sqrt{2}}(\ket{0}+e^{2\pi i (2^1 \varphi)}\ket{1})=\ket{\psi_1}$}\qw\\
    % ROW 5
    &\lstick{$\ket{0}$} & \gate{H} &
        \ctrl{1} & \qw & \qw & \ \ldots\ \qw & \qw & \rstick{$\frac{1}{\sqrt{2}}(\ket{0}+e^{2\pi i (2^0 \varphi)}\ket{1})=\ket{\psi_0}$}\qw\\
    % ROW 6
    \lstick[wires=1]{2nd Register:\\$m$ Qbits }
    &\lstick{$\ket{u}$} & \qwbundle[alternate]{} &
        \gate{U^{2^0}}\qwbundle[alternate]{} & \gate{U^{2^1}}\qwbundle[alternate]{} & \gate{U^{2^2}}\qwbundle[alternate]{} & \ \ldots\ \qwbundle[alternate]{} & \gate{U^{2^{t-1}}}\qwbundle[alternate]{} & \rstick{$\ket{u}$}\qwbundle[alternate]{}
    \end{quantikz}
}
\caption{The quantum circuit diagram for what Nielsen \& Chuang (2010) call the ``first stage'' of the quantum phase estimation algorithm (largely a replication of their Figure 5.2, pg 222).}
\label{fig:QPE_circuit_diagram}
\end{figure}

\clearpage

\item[\textbf{(\textit{d})}] Final state of 1st register at the end of 1st stage \ldots \ldots
  
  \vspace{0.05in}
  
  The final state of the 1st register at the end of this 1st stage is then:

  \begin{align}
      \label{eq:1st_stage_1st_register}
      &\frac{1}{\sqrt{2}}\left(\ket{0}+e^{2\pi i 2^{t-1}\varphi}\ket{1}\right)
      \otimes
      \frac{1}{\sqrt{2}}\left(\ket{0}+e^{2\pi i 2^{t-2}\varphi}\ket{1}\right)
      \otimes
      \cdots
      \otimes
      \frac{1}{\sqrt{2}}\left(\ket{0}+e^{2\pi i 2^1\varphi}\ket{1}\right)
      \otimes
      \frac{1}{\sqrt{2}}\left(\ket{0}+e^{2\pi i 2^0\varphi}\ket{1}\right)\\
      &=
      \frac{1}{2^{t/2}}\left(
      \ket{0}_{t}
      + e^{2\pi i \varphi}\ket{1}_{t}
      + e^{2\pi i (2\varphi)}\ket{2}_{t}
      + \cdots
      + e^{2\pi i ((2^t-1)\varphi)}\ket{2^t-1}_{t}
      \right)\\
      &=
      \frac{1}{2^{t/2}}
      \sum_{k=0}^{2^t-1}
      e^{2\pi i k\varphi}\ket{k}_{t}
  \end{align}
  
  where the subscript $t$ is a reminder we are working over a $t$-Qbit register.
  
\end{itemize}

%%%%%%%%%%%%%%%%%%%%%%%%%%%%%%%%%
%   N&C Exercise 5.7 (pg 222)   %
%%%%%%%%%%%%%%%%%%%%%%%%%%%%%%%%%

\item Intercalary: Nielsen \& Chuang's Exercise 5.7 (pg 222):
\vspace{0.1in}
\begin{adjustwidth}{100pt}{100pt}
\begin{mdframed}[hidealllines=true, backgroundcolor=gray!20]
\textbf{Exercise 5.7:} Additional insight into the circuit in Figure 5.2 may be obtained by showing, as you should now do, that the effect of the sequence of controlled-U operations like that in Figure 5.2 is to take the state $\ket{j}\ket{u}$ to $\ket{j}U^{j}\ket{u}$. (Note that this does not depend on $\ket{u}$ being an eigenstate of $U$.)
\end{mdframed}
\end{adjustwidth}

\textbf{Solution.} We take $\ket{j} = \ket{j}_{t}$ --- \textit{i.e.}, $j$ is an integer value expressible in $t$ bits, so that
\begin{align}
    j = \sum_{i=0}^{t-1} \alpha_i 2^i, \text{ for } \alpha_i \in \{0, 1\}
\end{align}
and thus
\begin{align}
    \ket{j} = \ket{\alpha_{t-1}}\ket{\alpha_{t-1}}\ldots\ket{\alpha_{1}}\ket{\alpha_{0}}
\end{align}
Then the effect of the sequence of controlled-$U$ operations like that in Figure 5.2 is to effect the transformation:
\begin{align}
    \ket{j}\ket{u} 
    &\mapsto
    \ket{j}\prod_{i=0}^{t-1}{e^{2\pi i (\alpha_i 2^i \varphi)}}\ket{u}
\end{align}
because for each non-zero $\alpha_i$ in the expansion of $j$, we will obtain a (non-unitary) factor of $e^{2\pi i (2^i \varphi)}$. The product of the exponential functions can be re-expressed as a sum in the exponent, where the sum is exactly our formula for $j$: 
\begin{align}
    \ket{j}\prod_{i=0}^{t-1}{e^{2\pi i (\alpha_i 2^i \varphi)}}\ket{u}
    &=
    \ket{j}{e^{2\pi i (\sum_{i=0}^{t-1}(\alpha_i 2^i) \varphi)}}\ket{u}\\
    &=
    \ket{j}{e^{2\pi i (j \varphi)}}\ket{u}\\
    &=
    \ket{j}{\left(e^{2\pi i \varphi}\right)^j}\ket{u}\\
    &=
    \ket{j}{U^j}\ket{u}\\
\end{align}
where in the end we used the fact that $U\ket{u} = e^{2\pi i \varphi}\ket{u}$ and thus $U^{j}\ket{u} = (e^{2\pi i \varphi})^j\ket{u}$.

Some simple concrete examples of this result are shown worked out from scratch in Table \ref{table:Prob_5.7} for $j$ defined on $t=3$ Qbits.

\begin{table}[th]
    \captionsetup{font=small, width = 14cm}
    \centering
    \begin{tabular}{c|c|c|c}
        $\ket{j}\ket{u}$ input & output & simplified output\\
        \hline\hline
        &&\\[-5pt]
        $\ket{001}\ket{u}$
        & $\ket{001}U^{2^0}\ket{u}$
        & $\ket{001}U^{1}\ket{u} = \ket{1}_3 U^{1}\ket{u}$
        \\[5pt]
        \hline
        &&\\[-5pt]
        $\ket{010}\ket{u}$
        & $\ket{010}U^{2^1}\ket{u}$
        & $\ket{010}U^{2}\ket{u} = \ket{2}_3 U^{2}\ket{u}$
        \\[5pt]
        \hline
        &&\\[-5pt]
        $\ket{011}\ket{u}$
        & $\ket{011}U^{2^1}U^{2^0}\ket{u}$
        & $\ket{011}U^{2}U^{1}\ket{u} = \ket{3}_3 U^{3}\ket{u}$
        \\[5pt]
        \hline
        &&\\[-5pt]
        $\ket{100}\ket{u}$
        & $\ket{100}U^{2^2}\ket{u}$
        & $\ket{100}U^{4}\ket{u} = \ket{4}_3 U^{4}\ket{u}$
        \\[5pt]
        \hline
        &&\\[-5pt]
        $\ket{101}\ket{u}$
        & $\ket{101}U^{2^2}U^{2^0}\ket{u}$
        & $\ket{101}U^{4}U^{1}\ket{u} = \ket{5}_3 U^{5}\ket{u}$
        \\[5pt]
        \hline
        &&\\[-5pt]
        $\ket{110}\ket{u}$
        & $\ket{110}U^{2^2}U^{2^1}\ket{u}$
        & $\ket{110}U^{4}U^{2}\ket{u} = \ket{6}_3 U^{6}\ket{u}$
        \\[5pt]
        \hline
        \end{tabular}
    \caption{Examples of results for Exercise 5.7, with $j$ defined on 3 Qbits.}
    \label{table:Prob_5.7}
\end{table}

%%%%%%%%%%%%%%%%%%%%%%%%%%%%%%%%%
%   Inverse Fourier Transform   %
%%%%%%%%%%%%%%%%%%%%%%%%%%%%%%%%%

\item 2nd Stage: Applying the Inverse Fourier Transform (and N\&C's Eqs 5.21–5.22)

\vspace{0.1in}

Suppose for the moment that our phase $\varphi$ can be represented exactly with our $t$ bits in the first register and so we can express our phase in binary decimal form as $\varphi = 0.\varphi_1 \varphi_2 \ldots \varphi_{t-1}\varphi_{t}$, where each $\varphi_{i}\in\{0, 1\}$ --- that is, the phase $\varphi$ can be expressed as:

\begin{align}
    \varphi
    &=
    \frac{\varphi_1}{2^1} + \frac{\varphi_2}{2^2}
    + \cdots + \frac{\varphi_{t-1}}{2^{t-1}}
    + \frac{\varphi_{t}}{2^{t}}\\
    &=
    \sum_{j=1}^{t} \varphi_{j} 2^{-j}
\end{align}

Then, as shown in N\&C's Eq 5.21 (pg 22), the first register state in expression (\ref{eq:1st_stage_1st_register}) resulting from the first stage of processing can be re-expressed as:
\begin{align}
&\begin{multlined}
    \frac{1}{2^{t/2}}\left(\ket{0}+e^{2\pi i 2^{t-1}(0.\varphi_1\varphi_2\ldots\varphi_{t})}\ket{1}\right)
      \left(\ket{0}+e^{2\pi i 2^{t-2}(0.\varphi_1\varphi_2\ldots\varphi_{t})}\ket{1}\right)
      \cdots\\
      \left(\ket{0}+e^{2\pi i 2^{1} (0.\varphi_{1}\varphi_2\ldots\varphi_{t})}\ket{1}\right)
      \left(\ket{0}+e^{2\pi i 2^0 (0.\varphi_1\varphi_2\ldots\varphi_{t})}\ket{1}\right)
\end{multlined}\\
&=
\begin{multlined}
    \frac{1}{2^{t/2}}\left(\ket{0}+e^{2\pi i (\varphi_1\varphi_2\ldots\varphi_{t-1}.\varphi_{t})}\ket{1}\right)
      \left(\ket{0}+e^{2\pi i (\varphi_1\varphi_2\ldots.\varphi_{t-1}\varphi_{t})}\ket{1}\right)
      \cdots\\
      \left(\ket{0}+e^{2\pi i (\varphi_{1}.\varphi_2\ldots\varphi_{t})}\ket{1}\right)
      \left(\ket{0}+e^{2\pi i (0.\varphi_1\varphi_2\ldots\varphi_{t})}\ket{1}\right)
\end{multlined}\label{eq:leading_digit_not_dropped}\\
&=
    \frac{1}{2^{t/2}}\left(\ket{0}+e^{2\pi i (0.\varphi_{t})}\ket{1}\right)
      \left(\ket{0}+e^{2\pi i (0.\varphi_{t-1}\varphi_{t})}\ket{1}\right)
      \cdots
      \left(\ket{0}+e^{2\pi i (0.\varphi_2\ldots\varphi_{t})}\ket{1}\right)
      \left(\ket{0}+e^{2\pi i (0.\varphi_1\varphi_2\ldots\varphi_{t})}\ket{1}\right)\label{eq:leading_digit_dropped}
\end{align}
where the leading digits are gradually ``dropped'' from the expression for $\varphi$ as we apply successive multiplications by $2$. Since it may not be obvious how/why we can simply drop those leading $\varphi_j$ digits with each successive multiplication by a factor of 2 in the exponential function, it is worth considering the step from the expression in (\ref{eq:leading_digit_not_dropped}) to expression (\ref{eq:leading_digit_dropped}), for the exponential function in just the second-to-last right-hand term:
\begin{align}
    e^{2\pi i 2^1 (0.\varphi_{1}\varphi_2\ldots\varphi_{t})}
    &=
    e^{2\pi i (\varphi_{1}.\varphi_2\ldots\varphi_{t})}\\
    &=
    e^{2\pi i (\varphi_{1} + 0.\varphi_2\ldots\varphi_{t})}\\
    &=
    e^{2\pi i (\varphi_{1})}
    e^{2\pi i ( 0.\varphi_2\ldots\varphi_{t})}\\
    &=
    (1)
    e^{2\pi i ( 0.\varphi_2\ldots\varphi_{t})}\\
    &=
    e^{2\pi i ( 0.\varphi_2\ldots\varphi_{t})}
\end{align}

In other words, since $e^{2\pi i \varphi_j} = 1$ whenever $\varphi\in \{0, 1\}$, those leading digits $\varphi_j$ each time just produce another factor of $1$ each time and can be removed.

\vspace{0.1in}

What's beautiful about the resulting expression in (\ref{eq:leading_digit_dropped}), though, is that it matches the quantum Fourier transform output shown in N\&C's Eq (5.4) (pg 218):

\begin{align}
    \ket{j_1, j_2, \ldots, j_n}
    \xrightarrow{QFT}
    \frac{1}{2^{t/2}}\left(\ket{0}+e^{2\pi i (0.j_n)}\ket{1}\right)
      \left(\ket{0}+e^{2\pi i (0.j_{n-1}j_{n})}\ket{1}\right)
      \cdots
      \left(\ket{0}+e^{2\pi i (0.j_1 j_2\cdots j_{n})}\ket{1}\right)
\end{align}

Thus applying the inverse quantum Fourier transform to our 1st-stage 1st-register result gives us:

\begin{align}
    \frac{1}{2^{t/2}}\left(\ket{0}+e^{2\pi i (0.\varphi_{t})}\ket{1}\right)
      \left(\ket{0}+e^{2\pi i (0.\varphi_{t-1}\varphi_{t})}\ket{1}\right)
      \cdots
      \left(\ket{0}+e^{2\pi i (0.\varphi_1\varphi_2\ldots\varphi_{t})}\ket{1}\right)
    \xrightarrow{QFT^{-1}}
    \ket{\varphi_1\varphi_2\ldots\varphi_{t}}
\end{align}

and thus (as Nielsen \& Chuang point out at the bottom of pg 222), a measurement in the computational basis gives us $\varphi$ exactly.

\clearpage

% \clearpage
\end{enumerate}

% \clearpage
% ========== %
% END MATTER %
% ========== %
% \newpage
\printbibliography

\end{document}